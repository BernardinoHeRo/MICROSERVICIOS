\noindent
Escribe el siguiente código en un fichero llamado \texttt{hello.c}:

\begin{lstlisting}[style=C]
	#include <stdio.h>
	int main(int argc, char* argv[]) {
		puts("Hola mundo!");
	}
\end{lstlisting}

\noindent
Ahora compila usando \texttt{gcc}:

\begin{listing}[style=consola, numbers=none]
	
	$ gcc  -o hello hello.c
	
\end{listing}

\noindent
Escribe el siguiente código en un fichero llamado \texttt{Hola.java}:

\begin{lstlisting}[style=C]
	public class Ejemplo {
		public static void main(String[ ] arg) {
			System.out.println("Hola Java");
		}
	}
\end{lstlisting}

\noindent
Ahora compila usando \texttt{javac}:

\begin{listing}[style=consola, numbers=none]
	
	$ javac ejemplo.java
	
\end{listing}